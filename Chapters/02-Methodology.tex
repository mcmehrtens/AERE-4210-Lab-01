\chapter{Methodology}\label{cp:methodology}

Using Ansys, the given simply-supported truss structure was analyzed. This was done by first, using the Ansys Mechanical APDL Product Launcher to launch Ansys and selecting the Use Shared-Memory Parallel option under High Performance Computer Setup. The analysis was initialized by selecting Structure as the preference. 

To begin preprocessing of the structures and properties, the 3D finite STN 180 type link element was added with a reference number of 1. Next, link sections were added, and the cross-sectional areas were defined. Link section 1 was named as base and had a link area of 0.00025 $m^{2}$, whereas link section 2 was named upper and had a link area of 0.00005 $m^{2}$. Material properties were added by selecting Material models, Structural, Linear, Elastic, and Isotropic. A Young's Modulus of 210 GPa and a PRXY of 0.3 was set for material 1, and a Young's Modulus of 70 GPa and a PRXY of 0.3 was set for material 2. Next, nodes were created in active CS with node 1 having an X,Y,X location of 0.0, 0.0, 0.0 m, node 2 located at 1.0, 0.0, 0.0 m, node 3 at 2.0, 0.0, 0.0 m, and node 4 at 1.0, 0.75, 0.0 m. Element attributes were set with members 1-2 and 2-3 being of element type 1 LINK180 and material and section numbers of 1 and members 1-4, 2-4, and 3-4 being of element type 1 LINK180 and material and section numbers of 2. Lastly, node and element numbers were turned on before all model components were checked. 

The next preprocessing step was to apply the appropriate loads to the structure. This was done by first defining the loads as structural displacements on the nodes. Node 1 was constrained in the X, Y, and Z directions, node 2 was constrained in the Y and Z directions, and nodes 2 and 4 were constrained only in the z direction. Then, a constant structural force of 1500 N was applied to node 4 in the negative Y direction. All of the constraints and forces were checked before analysis. 

To compute the numerical solution, a new static analysis was selected, solved, and reviewed. The display background was changed to white so that the results could be viewed more clearly. Then, the deformed and undeformed shapes were plotted together and saved as a .tif file. By selecting the List Results setting, the nodal solution's displacement vector sum and all the reaction solution's data were saved as text files. The element table was then defined to list the stresses and element forces. To get the element stress, the item MemStress was created, and the results were computed by sequence number LS,1. To get the element force, the item MemForce was created, and the results were computed by sequence number SMISC,1. To get the element volume, the item MemVol was created and computed using geometry. Lastly, the element table was listed, and MemStress, MemForce, and MemVol were saved separately and as a combined table.