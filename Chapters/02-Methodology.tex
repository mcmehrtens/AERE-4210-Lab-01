\chapter{Methodology}\label{cp:methodology}

Using Ansys, the simply-supported truss structure defined by \autoref{fig:truss} and \autoref{tab:truss} was analyzed. To begin the analysis, the Ansys Mechanical APDL Product Launcher was used to launch Ansys with the ``Use Shared-Memory Parallel'' option selected under ``High Performance Computer Setup.'' The analysis then began by selecting ``Structure'' from the ``Preferences'' menu.

To begin preprocessing the structure, the ``3D finite STN 180'' link was added with a reference number of \num{1}. Next, link sections were added, and the cross-sectional areas were defined. Link section \num{1} was named ``base'' and had a link area of \qty{0.00025}{\meter^2}. Link section \num{2} was named ``upper'' and had a link area of \qty{0.00005}{\meter^2}. After the link sections were added, the material properties were defined. A Young's modulus of \qty{210}{\giga\pascal} and a ``PRXY'' of \num{0.3} was set for material \num{1} and a Young's modulus of \qty{70}{\giga\pascal} and a ``PRXY'' of \num{0.3} was set for material \num{2}. Next, nodes were created in the active coordinate system with node \num{1} located at $\left(\qty{0.0}{\meter},\qty{0.0}{\meter},\qty{0.0}{\meter}\right)$, node \num{2} located at $\left(\qty{1.0}{\meter},\qty{0.0}{\meter},\qty{0.0}{\meter}\right)$, node \num{3} at $\left(\qty{2.0}{\meter},\qty{0.0}{\meter},\qty{0.0}{\meter}\right)$, and node \num{4} at $\left(\qty{1.0}{\meter},\qty{0.75}{\meter},\qty{0.0}{\meter}\right)$. Element attributes were configured with members \numrange[range-phrase = --]{1}{2} and \numrange[range-phrase = --]{2}{3} set to element type ``1 LINK180'', material number \num{1}, and section number \num{1}. Members \numrange[range-phrase = --]{1}{4}, \numrange[range-phrase = --]{2}{4}, and \numrange[range-phrase = --]{3}{4} were set to element type ``1 LINK180'', material number \num{1}, and section numer \num{2}. Lastly, node and element numbers were turned on before all model components were verified. 

The next preprocessing step was to apply the loads to the structure. This was done by first adding displacement constraints on the nodes. Node \num{1} was constrained in the $x$, $y$, and $z$ directions; node \num{2} was constrained in the $y$ and $z$ directions; and nodes \numlist{2;4} were constrained in the $z$ direction. Then, a constant structural force of \qty{1500}{\newton} was applied to node \num{4} in the negative $y$ direction. All of the constraints and forces were checked before running the analysis. 

To compute the numerical solution, a new static analysis was selected, solved, and reviewed. The display background was changed to white so that the results could be viewed more clearly. Then, the deformed and undeformed shapes were plotted together and saved as a \verb|.tif| file. By selecting the ``List Results'' setting, the nodal solution and the reaction solution were saved as text files. The element table was then defined to list the stresses and element forces. To get the element stress, the item ``MemStress'' was created, and the results were computed by sequence number ``LS,1.'' To get the element force, the item ``MemForce'' was created, and the results were computed by sequence number ``SMISC,1.'' To get the element volume, the item ``MemVol'' was created and computed using geometry. Lastly, the element table was listed, and ``MemStress'', ``MemForce'', and ``MemVol'' were saved separately and as a combined table. All the tables saved can be found in \autoref{sec:ansys_output}.